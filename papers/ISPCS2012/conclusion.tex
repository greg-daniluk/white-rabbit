\section{Conclusions}
\label{sec:conclusions}

In this paper the first deployment of a ``beta version" of a White Rabbit
system is described. The deployed system includes a WR Network (consisting of switches) 
interconnecting WR Nodes. 

The results indicate that a system based on the 
White Rabbit technology is capable of providing nanosecond accuracy of synchronization 
over large distances (i.e. over 16 km). The measured accuracy of the deployed system  
is 0.517~ns and the precision is 0.119~ns. The calculated MTIE is below 1.05~ns
with only 0.0003$\%$ of values exceeding the $\pm$0.5~ns range.
The WR-timebase 
guarantees sub-nanosecond accuracy and tens of picoseconds precision of the distributed time 
and frequency reference regardless of the changing temperature conditions. 
The standard deviation of the skew measured between the time reference (grandmaster) and 
the nodes (over a peak-to-peak 45 
degrees Celsius temperature range) is 55~ps while the MTIE is below 342~ps. 
The temperature tests indicate that the acceptable influence   
of the temperature variation of WR devices on the quality of synchronization can be easily
reduced by compensating temperature-induced changes of the hardware delays. 
Such a compensation should be considered in future developments of the WR technology.

The high accuracy and precision time transfer over an Ethernet-based WR Network has many potential
applications. Precise time-tagging of the input events using WR-provided timebase is the first to
be realized. Therefore, the described deployment marks an important milestone in the White Rabbit Project -- 
proof-of-concept technology becomes a working solution. This solution is about to be 
commercially available while sustaining its openness (open hardware and open software). 


\newpage