\begin{abstract}
%\boldmath

This paper describes the long-term performance of White Rabbit (WR) based
time and frequency transfer in the systems deployed at CERN and Gran 
Sasso National Laboratory. WR is a new technology based on 
IEEE 1588-2008 and Synchronous Ethernet which allows for sub-nanosecond 
accuracy and picosecond precision of synchronization in 
\modified{an Ethernet-based network.}
%the entire WR network.
The first installation of WR is used in the CERN Neutrino 
to Gran Sasso (CNGS) project to transfer the Coordinated Universal Time 
(UTC) from a Global Positioning System (GPS) receiver to the underground 
extraction/detection points. The data collected during the system operation 
is used to evaluate its performance. Additionally, the performance
in varying temperature conditions is verified with tests in a climatic chamber. 
We evaluate time transfer \modified{by} measuring the offset between the time 
reference and the time receiver (WR Node). 
% We first provide a short introduction to WR and the underlying 
% technologies. Then, we describe the CNGS project focusing on the WR installation 
% and measurement setup. Finally we analyze the collected data. We evaluate 
% time transfer measuring the standard deviation of the offset between the time 
% reference (cesium clock) and the time receiver (WR Node). 
The stability of the transfered frequency is evaluated \modified{by} %measuring phase noise and 
analyzing the Allan Deviation (ADEV) and the Maximum Time Interval Error (MTIE). 

\end{abstract}


