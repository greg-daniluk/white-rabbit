\documentclass{../JAC2003}

\addtolength{\topmargin}{-18mm} % add this line for JAC2003

%% This file was updated in March 2011 by T. Satogata to be in line
%% with Word templates.
%%
%%  Use \documentclass[boxit]{JAC2003}
%%  to draw a frame with the correct margins on the output.
%%
%%  Use \documentclass[acus]{JAC2003}
%%  for US letter paper layout
%%

\usepackage{graphicx}
\usepackage{booktabs}
\usepackage{textcomp} % for \textmu (greek letter mu without the need
                      % to use Math mode)

%%
%%   VARIABLE HEIGHT FOR THE TITLE BOX (default 35mm)
%%

%\setlength{\titleblockheight}{27mm} % this was the default
\setlength{\titleblockheight}{30mm}

\begin{document}

\title{THE WHITE RABBIT PROJECT}

\author{J. Serrano, M. Cattin, E. Gousiou, E. van der Bij,
  T. W\l{}ostowski, CERN, Geneva, Switzerland\\
  G. Daniluk, AGH University of Science and Technology, Krakow, Poland\\
  M. Lipi\'{n}ski, Warsaw University of Technology, Warsaw, Poland}

\maketitle

\begin{abstract}

  White Rabbit (WR) is a multi-laboratory, multi-company collaboration
  for the development of a new Ethernet-based technology which ensures
  sub-nanosecond synchronisation and deterministic data transfer. The
  project uses an open source paradigm for the development of its
  hardware, gateware and software components. This article provides an
  introduction to the technical choices and an explanation of the
  basic principles underlying WR. It then describes some possible
  applications and the current status of the project. Finally, it
  provides insight on current developments and future plans.

\end{abstract}

\section{INTRODUCTION}

The White Rabbit (WR) project~\cite{white-rabbit-ref} was initiated at
CERN in 2008 to start preparing the evolution of the General Machine
Timing (GMT) system. The GMT is based on uni-directional 500 kb/s
RS422 links, and allows operators and other users to synchronise
different processes in CERN's accelerator network. The system has a
number of shortcomings though, among which the most important are the
limited bandwidth and the impossibility of dynamically evaluating the
delay induced by the data links.

WR started with the following specifications:
\begin{Itemize}
 \item Transfer of a time reference from a central location to many
   destinations with an accuracy better than 1~ns and a precision
   better than 50~ps.
 \item Ability to service more than 1000 nodes.
 \item Ability to cover distances of the order of 10~km.
 \item Data transfer from a central controller to many nodes with a
   guaranteed upper bound in latency.
\end{Itemize}

One of the main aims of the project is to deliver the above
functionality while using --~or extending where needed~-- existing
standards. Ethernet was chosen as the physical layer for all data
transmission. The most precise standard synchronisation method for
Ethernet networks is the Precise Time Protocol (PTP), standardised as
IEEE~1588. With PTP it is possible to synchronise stations with an
accuracy in the order of 1~ \textmu s. WR extends PTP in a
backwards-compatible way to achieve sub-ns accuracy.

Fig.~\ref{hierarchy-fig} shows the layout of a typical WR
network. Data-wise it is a standard Ethernet switched network,
i.e. there is no hierarchy. Any node can talk to any other
node. Regarding synchronisation, there is a hierarchy established by
the fact that switches have downlink and uplink ports. A switch uses
its downlink ports to connect to uplink ports of other switches and
discipline their time. The uppermost switch in the hierarchy receives
its notion of time through external TTL Pulse Per Second (PPS) and
10~MHz inputs, along with a time code to initialise its internal
International Atomic Time (TAI) counter.  

\begin{figure}[htb]
   \centering
   \includegraphics*[width=\columnwidth]{../../figures/network/hierarchy.eps}
   \caption{Layout of a typical WR network.}
   \label{hierarchy-fig}
\end{figure}

WR switches allow users to build highly deterministic data networks by
having different internal queues for Ethernet frames of different
priorities, as established by the priority header defined in IEEE
802.1Q. The combination of deterministic latencies and a common notion
of time to within 1~ns allows WR to be a suitable technology to solve
many problems in distributed real-time controls and data
acquisition. The following section describes the technologies used to
cope with the synchronisation requirements in WR.

\section{TECHNOLOGIES}

The three key technologies used in WR to achieve sub-ns accuracy in
synchronisation are PTP, layer-1 syntonization and precise phase
measurements. In the following paragraphs, we describe each one in
turn.

\subsection{Precise Time Protocol}

The IEEE 1588 standard specifies a way to evaluate the link delay
between two nodes --~one master and one slave~-- through the exchange
of time-tagged messages. Fig.~\ref{ptp-fig} shows a simplified view of
these exchanges.

\begin{figure}[htb]
   \centering
   \includegraphics*[width=0.5\columnwidth]{../../figures/protocol/ptpMSGs-color.eps}
   \caption{Simplified PTP message exchange diagram.}
   \label{ptp-fig}
\end{figure}

The master node sends an initial message to the slave and stamps it with
a time-stamp $t_1$ as it goes out of its networking interface. The
message is received at a time $t_2$ in the slave's time base. The
process is then reversed, with a new message sent from the slave at
time $t_3$ and received in the master at time $t_4$. Assuming that the
one-way delay through the network is exactly half of the two-way delay
--~an assumption which is never completely accurate~-- the one-way
delay can be estimated as:

\begin{equation}\label{ptp-eq}
    \delta={{(t_4 - t_1) - (t_3 - t_2)} \over 2} 
\end{equation}

Typical PTP implementations use free-running oscillators in each
node. This means that there will be a growing time drift between
master and slave unless the message exchange and calculation of
$\delta$ happen repeatedly. Even if there is such a continuous
exchange of messages, the time bases will drift during the time
interval between two calculations of $\delta$. WR nodes extract the
clock signal from the incoming data stream, through a mechanism called
``layer-1 syntonization''. This results in equal clock frequencies in
all nodes, therefore eliminating the drift problem present in typical
PTP implementations.

\subsection{Layer-1 Syntonization}

In WR, as in the ITU-T Synchronous Ethernet standard, the mechanism
used to guarantee that all nodes are clocked at the same frequency
works at the level of the physical layer, with no impact whatsoever on
data traffic. WR currently supports Gigabit Ethernet (GbE) on fibre
only. In GbE, the link is never idle. Even if a node has nothing to
transmit, its Medium Access Control (MAC) block generates special
8B10B patterns called commas, whose purpose is to avoid unlocking of the Phase
Locked Loop (PLL) on the Clock and Data Recovery (CDR) circuit in the
receive (RX) path of the node connected to it. Commas also help in aligning the
serial-to-parallel converter on the receiving node. 

Fig.~\ref{synce-fig} depicts the mechanism of layer-1
syntonization. In standard Ethernet, each node uses its own
free-running clock to encode the messages it sends to the node on the
opposite side of the link. By contrast, a network using layer-1
syntonization establishes a clocking hierarchy, whereby there is one
master node or switch. All other nodes and switches extract their
clock signals from data streams, in such a way that all of the system
clocks of nodes and switches on the network end up beating at exactly
the same rate. Switches play a key role in this frequency
distribution, by extracting the clock from the data stream going into
one of their ports (uplink port) and using that extracted clock in the
encoding of all data streams going out of all ports (uplink and
downlinks).

\begin{figure}[htb]
   \centering
   \includegraphics*[width=\columnwidth]{../../figures/misc/eth_vs_synce.eps}
   \caption{Simple illustration of layer-1 syntonization.}
   \label{synce-fig}
\end{figure}

\subsection{Precise Phase Measurement}

As we saw in the preceding paragraph, a downlink port of a switch
disciplines the frequency of a downstream switch by connecting to its
uplink port. The downstream switch, in turn, uses this extracted clock
to encode the data which it sends back to the upstream switch. In this
way, the upstream switch finds a delayed copy of its encoding clock in
the output of the CDR circuit in its RX path, as depicted in
Fig.~\ref{phase-tracking-fig}. The phase shift between these two clock
signals is directly related to the link delay, and a measurement of
this phase difference can therefore be incorporated into the PTP
equation in order to achieve better precision. 

\begin{figure}[htb]
   \centering
   \includegraphics*[width=\columnwidth]{../../figures/misc/phase_tracking.eps}
   \caption{Phase tracking block diagram.}
   \label{phase-tracking-fig}
\end{figure}

In addition, a phase-shifting circuit can be included in the slave
node to create a phase-compensated clock signal, i.e. a clock signal
which is in phase with the master clock signal despite the delay
introduced by the fibre link. The delay programmed into this phase
shifter at any given time is of course taken into account when
calculating the link delay.

The introduction of layer-1 syntonization and phase tracking provides
a synchronisation mechanism which can potentially be completely
independent of the data link layer. Beyond the first PTP message
exchange, there is in principle no need to continue exchanging
messages to keep the nodes synchronised. In practice, WR keeps the PTP
messages going for robustness reasons, albeit at a much reduced
rate. This makes the PTP traffic negligible in terms of bandwidth,
therefore not getting in the way of determinism for the user frames,
another key requirement for WR. 

Layer-1 syntonization also allows to cast the time-stamping problem
into a phase measurement problem, which increases precision since
phase measurement can be done much more precisely than measuring time
intervals. The circuit we use for measuring phases, called Digital
DMTD (Dual Mixer Time Difference), is depicted in
Fig.~\ref{ddmtd-fig}.

\begin{figure}[htb]
   \centering
   \includegraphics*[width=\columnwidth]{../../figures/misc/dmtd2.eps}
   \caption{Digital DMTD circuit.}
   \label{ddmtd-fig}
\end{figure}

The flip-flops (FFs) in this digital implementation play the role of
the mixers in the original DMTD circuit~\cite{dmtd-ref}. This circuit
measures the phase difference between two clock signals, $clk_A$ and
$clk_B$, which have the same nominal frequency. A PLL is used to
generate a third clock signal, very close but not quite at the same
frequency as $clk_A$ and $clk_B$. The FFs sample the two incoming
signals with this synthesised clock signal. Since the sampling
frequency is very close to that of the incoming signals, very low
frequency waveforms result at the outputs of the FFs. There are some
glitches in the edges of these low-frequency square waves as a result
of the very slow sweeping and the jitter in the incoming signals. The
slow sweeping --~equivalent to the low-frequency beat in the analogue
version of the circuit~-- provides a magnifying effect. Tiny phase
differences in the input signals become readily measurable time
intervals after sampling and cleaning up the glitches. The circuit is
fully digital and extremely linear. This circuit is used as a phase
detector in the master and as part of the phase shifter in the
slave. Indeed, one can build a very linear phase shifter by using this
phase detector in a loop, and having a non-linear phase shifter shift
the phase until the linear phase detector signals the nominal phase
shift is attained.

\section{THE WHITE RABBIT SWITCH}

WR is a switched network. At its heart lies its most important
component: the WR switch, which provides 18 ports in a 1U 19''
rackable enclosure. It is made of open source hardware, gateware and
software, and it is sold and supported by a commercial company. WR
switches are fully compatible with Ethernet, and can identify if a WR
node or another WR switch is hooked to one of their ports by using the
WR extension~\cite{wr-extension-ref} to the IEEE 1588 protocol at
link establishment time. This extension is also designed to be
backwards-compatible with standard PTP, so it is possible to connect
existing PTP gear to a network made with WR switches, along with WR
nodes. In this case, the WR nodes will benefit from the extension and
therefore achieve better accuracy, while the standard PTP nodes will
run only the standard protocol and feature reduced accuracy.

\subsection{Architecture}

Fig.~\ref{switch-top-fig} shows a high-level block diagram of the WR
switch. Ethernet frames are exchanged through 18 ports equipped with
Small Form-factor Pluggable (SFP) sockets which can host optical
transceivers. The reference implementation uses SFP modules for one
single mode fibre, using one wavelength for TX traffic and a different
one for RX. The use of a single fibre ensures that the symmetry in TX
and RX paths --~after mathematically compensating for fibre
dispersion~-- is robust, in particular against changes in cabling not
notified to the final user. The SFPs are connected directly to a
Xilinx Virtex-6 Field Programmable Gate Array (FPGA).

\begin{figure}[htb]
   \centering
   \includegraphics*[width=0.7\columnwidth]{../../figures/switch/switch_simple_diagram.eps}
   \caption{High-level block diagram of the WR switch.}
   \label{switch-top-fig}
\end{figure}

Ethernet frames get switched inside the FPGA with very low latency. An
ARM CPU running Linux helps with less time-sensitive processes like
remote management and keeping the frame filtering database in the FPGA
up to date. The clocking resources block contains PLLs for cleaning up
and phase-compensating the system clock, as well as for generating the
frequency-offset DMTD clock.

\subsection{FPGA Design} 

Fig.~\ref{switch-fpga-fig} shows a block diagram of the internals of
the FPGA in the switch. Blocks under development are shown in white
boxes with grey borders, and will be discussed in another section. The
current release, without those blocks, is fully operational. The
design consists of Hardware Description Language (HDL) cores around a
Wishbone bus~\cite{wishbone-ref} interconnect.

\begin{figure}[htb]
   \centering
   \includegraphics*[width=\columnwidth]{../../figures/switch/switch_hdl_v3.3.eps}
   \caption{Block diagram of the FPGA design in the WR switch.}
   \label{switch-fpga-fig}
\end{figure}

There are 18 Endpoint blocks, each connected to an SFP in the
switch. They send and receive Ethernet frames using the switching core
(SwCore) to communicate with one another. The decision as to where a
frame is destined is taken by a forwarding process in the Routing
Table Unit (RTU). The choice of the word ``routing'' here is a bit
unfortunate since we are speaking about layer-2 forwarding, not
layer-3 routing. 

All traffic to and from the ARM CPU goes through the CPU EBI/WB
bridge. This interface is e.g. used to keep the database up to date in
the RTU. The ARM CPU itself can be a source or sink for Ethernet
frames, using the Network Interface Card (NIC) and the Vector
Interrupt Controller (VIC) blocks in the FPGA. The ARM CPU also runs
the PTP stack for the switch, and reads the time stamps for outgoing
frames from the TX Time Stamp Unit (TX TSU).

The Real Time (RT) subsystem block is responsible for timekeeping in
the switch. It contains an LM32 soft CPU running a soft PLL which
controls the various programmable oscillators in the switch. In
addition, it hosts the counters for the current TAI.

The Hardware Info Unit (HWIU) contains important information about the
global HDL design itself, such as the HDL commit hash in the Git
repository and the date of the synthesis. The I2C, GPIO and PWM blocks
control other peripherals on the switch, such as LEDs and the cooling
fans. 

\subsection{Performance}

In order to characterise the performance of the WR switches, a system
was set up in a laboratory consisting of four cascaded WR
switches. The master switch was connected to a first slave switch
through a 5~km fibre roll. Similar fibre rolls were used to connect
the first switch to the second one, and then the second one to the
third one, for a total of 15~km of fibre. Adverse conditions were
simulated by heating the fibre rolls with a hot air gun. 

Since the four switches were all in the same laboratory, it was easy
to monitor their PPS outputs with an oscilloscope and draw histograms
of the offsets between the PPS output in each switch and the PPS
output in the master switch. The results of these measurements can be
seen in Fig.~\ref{performance-fig}.

\begin{figure}[htb]
   \centering
   \includegraphics*[width=\columnwidth]{../../figures/measurements/meas_results2.eps}
   \caption{Histograms of PPS output offsets of three cascaded WR
     switches with respect to the PPS pulse output in the master switch.}
   \label{performance-fig}
\end{figure}

As can be seen in the plots, accuracy always stays within
$\pm$~200~ps, and typical precisions are in the order of 6~ps. The
same type of accuracy and precision is found with WR nodes, since the
technologies involved for delay measurement and compensation are
exactly the same as those used in the switches.

\section{WHITE RABBIT NODES}

Fig.~\ref{node-fig} shows a simplified block diagram for a WR node
based on the Simple PCI Express Carrier (SPEC)
board~\cite{spec-ref}. This card can host mezzanines conforming to the
FPGA Mezzanine Card (FMC) VITA 57 standard. We have developed Analogue
to Digital Converter (ADC), Time-to-Digital Converter (TDC) and
programmable delay generator FMCs. By plugging these cards in a
WR-enabled carrier such as the SPEC, and appropriately configuring the
FPGA in the carrier board, one can enhance their functionality with
features such as synchronous sampling clocks in remote nodes and
precise TAI time stamps.

\begin{figure}[htb]
   \centering
   \includegraphics*[width=\columnwidth]{../../figures/node/wrNode.eps}
   \caption{An example WR node.}
   \label{node-fig}
\end{figure}

In order to enable users to easily build nodes for WR-based
applications, we have developed a core which takes care of all WR data
transmission and reception, along with all synchronisation tasks. This
WR PTP Core (WRPC)~\cite{wrpc-ref} --~an Ethernet Medium Access
Control (MAC) unit with enhancements for timing~-- contains an LM32
soft CPU inside which runs the whole PTP stack. Frames which are
identified as non-PTP are forwarded downstream to the user
logic. Conversely, the core also accepts frames from user logic, which
can for example be used to stream data acquired in an ADC FMC. The
WRPC takes care of controlling the programmable oscillators on the
SPEC or any other WR-enabled board. An Etherbone slave
core~\cite{etherbone-ref} can optionally be instantiated between the
WRPC and the user logic. Etherbone is an independent project led by
GSI which can work in conjunction with any Ethernet MAC core, not
necessarily with the WRPC. It aims at providing a way to trigger reads
and writes in a remote Wishbone bus through carefully defined payloads
in an Internet Protocol (IP) packet. With Etherbone, a complete
network of sensors and actuators looks like a big memory map to a
master/management node.

WR-capable nodes have been designed in PCIe, PXIe, VME64x and \textmu
TCA form factors. These designs have varying degrees of maturity and
commercial support. The most mature one is the PCIe SPEC board, and
different WR-based gateware designs have been successfully targeted
at it, including a Network Interface Card with a Linux network device
driver.

\section{APPLICATION EXAMPLES}

WR technology provides users with a common notion of TAI in every node
and with a deterministic network in which an upper bound for latencies
is guaranteed by design. This opens up many possibilities in different
fields, such as Multiple Input Multiple Output (MIMO) feedback
systems. In this section, we describe just two of the multiple
possible applications of WR.

\subsection{RF Distribution}

WR distributes a 125 MHz clock --~typically TAI-related~-- for free. A
WR user gets access to this clock, or derivatives of it, just by
hooking a WR node to a WR network. However, in some cases users care
more about synchronising to Radio Frequency (RF) signals related to
e.g. the accelerating structures in a particle accelerator. Generating
phase-compensated RF signals in different locations can be useful in
other domains as well, such as in radar applications.

Fig.~\ref{d3s-fig} shows a block diagram of how one can use a WR network
to distribute RF clocks, through a scheme called Distributed Direct
Digital Synthesis (Distributed DDS or D3S).    

\begin{figure}[htb]
   \centering
   \includegraphics*[width=\columnwidth]{../../figures/applications/distributed_dds.eps}
   \caption{Distributed DDS in a WR network.}
   \label{d3s-fig}
\end{figure}

The reference clock line in the drawing is just conceptual. Nodes get
the reference clock from the WR network itself. There is no need for
additional connections. The transmitting node tracks an RF signal
connected to its input with a PLL in which the role of the
voltage-controlled oscillator is fulfilled by a DDS block. The control
words for that DDS, along with the TAI at which they were applied, are
encoded and broadcast through the WR network. Receiving nodes can then
apply a fixed offset to the TAI stamps and replay the RF waveform with
their local DDS blocks, with just a fixed delay. In most situations
the RF is stable enough for this fixed delay to be of no concern.

This scheme has several advantages over traditional RF distribution
systems. There is no additional cabling to be done. The same network
can handle more than one distributed RF. In addition, all waveforms
are played using a TAI-related clock, which is very useful for
diagnostics. A first crude implementation has been demonstrated at
CERN, with a jitter --~defined here as the integral of the Power
Spectral Density of the phase noise, integrating between 10~Hz and
5~MHz~-- in the replayed RF below 10~ps. Better jitter can be achieved
by carefully tuning the digital PLL filter and cleaning the output of
the DDS with an analogue PLL including a low phase noise oscillator.

\subsection{Distributed Oscilloscope}

Fig.~\ref{oscilloscope-fig} shows a conceptual representation of a
distributed oscilloscope using several of the building blocks we have
described so far. 

\begin{figure}[htb]
   \centering
   \includegraphics*[width=\columnwidth]{../../figures/applications/distributed_daq.eps}
   \caption{Distributed oscilloscope using a WR network.}
   \label{oscilloscope-fig}
\end{figure}

ADC nodes sample analogue signals synchronously in remote
locations. The synchronous phase-compensated sampling is facilitated
by the WR-derived clock. The ADCs can store their samples in rolling
buffers, where each location is known to contain the sample
corresponding to a precise TAI. As soon as an ADC node detects a
condition upon which it should trigger, it can broadcast a trigger
message through the WR network. All nodes will have received that
message after a guaranteed worst-case delay. These nodes can then stop
sampling and rewind their buffers to the TAI specified in the
triggering message. External trigger pulses can also be accommodated
through the use of TDC nodes, and sampling with non-TAI-related clocks
can be done using D3S nodes.

A computer in the control room then gets all the TAI-stamped
acquisition data and displays it coherently on a single screen. The
operators then see this whole distributed system as a simple
oscilloscope to which all their signals are connected. This would of
course be impossible with a real oscilloscope, but it can be done with
WR's phase-compensated distribution of TAI, appropriate WR-enabled
nodes and software. 

\section{THE WR COMMUNITY}

The WR project is a distributed endeavour whose collaborative model is
heavily inspired by that typically used in Free/Open Source software
projects. All the intermediate and final results in the project are
made available through open source licences. We use mostly the GNU
General Public License (GPL) and the GNU Lesser General Public License
(LGPL) for licensing software, LGPL for gateware (HDL files) and the
CERN Open Hardware Licence~\cite{cernohl-ref} for hardware designs.

WR users are very often developers and vice versa. The ever-growing
list of interested institutes and companies~\cite{wr-users-ref} has
enlarged the scope of the project well beyond the original domain of
particle accelerators. At CERN, the effort of dissemination for this
technology has been coordinated in collaboration with the Knowledge
Transfer Group.

Thanks to its open nature, the project has benefited from extensive
contributions and ideas of potential and confirmed users. These
exchanges typically happen on the project mailing list, which is
web-archived and integrated with the project
website~\cite{white-rabbit-ref}. Peer review over email or in
dedicated workshops establishes a meritocracy in which the best ideas
move forward. Openness is also key in avoiding potential vendor
lock-in situations and welcoming commercial partners as any other
contributor. 

\section{PROJECT STATUS AND OUTLOOK}

At the time of this writing (August 2013) the WR switch is a mature
commercially-supported hardware product with a stable release for its
gateware and software. WR nodes have been designed and validated in a
variety of formats, and they have consistently shown sub-ns
synchronisation accuracy. This section provides a quick overview of
some of the most important upcoming efforts in the project.

\subsection{Synchronisation Performance}

The results in Fig.~\ref{performance-fig} show that the original goals
for WR in terms of synchronisation have been achieved. In some
applications though, it is important to improve accuracy as much as
possible. While the WR delay model~\cite{wr-spec-ref} and compensation
mechanism cover changes in fibre temperature appropriately, they do
not compensate for the delay fluctuations induced by temperature
variations in the nodes and switches themselves. We believe these
effects can account for tens of picoseconds or even more in
particularly adverse circumstances.

Tests in a climatic chamber~\cite{torture-ref} have shown that the
dependency of delay with respect to temperature is mostly linear, so
an evolution of the WR delay compensation algorithm which takes into
account on-board temperature measurements should help in this
respect. The modified algorithm could include a linear model of the
dependency or a table of delay vs. temperature values resulting from
offline calibration.

\subsection{Switch Evolution}
% SNMP remote management
% mention PPSi here
% refer to figure for missing blocks in switch gateware

The efforts on switch gateware and software will go in the direction
of providing better support for remote management, diagnostics and
robust data transmission in a WR network. 

A Port Statistics block (Pstats, see Fig.~\ref{switch-fpga-fig}) will
be added to provide counts of transmitted, received and dropped frames
per port, along with other counts relevant for ascertaining the state
of health of the network. A Time-Aware Traffic Shaper Unit (TATSU)
will allow blocking selected output queues for some time so that
high-priority frames can go through the switch without colliding with
low-priority frames being sent at that moment through a port. This
will avoid having to wait until the end of transmission of those
frames before taking ownership of the port.

In addition, a Topology Resolution Unit (TRU) will add hardware
support to the process of providing redundant loop-free topologies for
a network. This will result in a faster topology switch-over compared
to traditional software-based methods such as the Rapid Spanning Tree
Protocol (RSTP). Fig.~\ref{fec-fig} illustrates why fast switch-over
can be important in a WR network.

\begin{figure}[htb]
   \centering
   \includegraphics*[width=\columnwidth]{../../figures/robustness/FEC.eps}
   \caption{Forward Error Correction used to correct for frame loss.}
   \label{fec-fig}
\end{figure}

In this example, a WR node decomposes a control message into four
Ethernet frames which have been encoded using Forward Error Correction
(FEC), in such a way that receiving any two of those frames allows the
receiving node to re-constitute the original control message. The use
of this type of FEC scheme is well adapted to networks in which a late
frame is a wrong frame, such as the one used for accelerator timing at
CERN. This precludes the use of protocols which require re-trials in
case of transmission errors, such as the Transmission Control Protocol
(TCP).

If the switch can enable a redundant port very quickly after detecting
a fatal condition in another port, and if the switch-over time is
lower than that corresponding to the transmission of one of the frames
in the figure, the receiving node will be able to re-constitute the
message with the remaining frames. Tests at CERN have shown that this
kind of switch-over speeds are indeed possible with appropriate
hardware support.

On the software front, most of the activity will be focused on
providing the switch with good Simple Network Management Protocol
(SNMP) support so that all control and diagnostics can happen remotely
using standard switch management software. Another important
development will be the replacement of the current PTP stack running
in the ARM processor by the PTP Ported to Silicon (PPSi)
stack. PPSi~\cite{ppsi-ref} is a portable PTP daemon which can be
targeted at bare-metal systems, such as the LM32 processor inside the
WRPC in the WR nodes, but can also run under an operating system, such
as the Linux running in the ARM processor in the switch. PPSi has been
developed in the frame of the WR project but can be used in any
project requiring PTP support. It is licensed under LGPL. 

\subsection{Standardisation}

Using standards is good for many reasons. A standard is more likely to
be used for a long time, so using it reduces long-term
risks. Standardisation bodies typically invest a big effort in making
sure standards are robust, so adopting them also saves time. Finally,
companies are more inclined to participate in a development project if
it is based on standards, because markets are typically larger in that
case and also because the rules of governance and evolution of the
standard are clear from the outset.

In WR, we are developing functionality which is not made available by
any existing standard. However, it was felt from the beginning of the
project that WR ideas could constitute the basis for the evolution of
the IEEE 1588 standard. The original WR specification was already
worded with this in mind. Now the IEEE P1588 Working
Group~\cite{ptp-wg-ref} has opened the process for the periodic
revision of the standard, and the WR project is represented in the
subcommittee for high accuracy enhancements. The aim of this effort
for the WR team is to end up in a situation where WR is just a
particularly accurate and precise implementation of the IEEE 1588
standard. The work of the subcommittee has just started, and the
outlook is very promising.

WR is also concerned with determinism, and the WR team keeps an eye on
the efforts of the Time Sensitive Networks (TSN) Task
Group~\cite{tsn-ref} inside the IEEE 802.1 Working Group. The ideas
discussed in this Task Group are of great relevance to WR. Conversely,
the TRU and TATSU blocks in Fig.~\ref{switch-fpga-fig} can be a very
convenient testing ground for these ideas. 

\section{ACKNOWLEDGEMENT}

The authors acknowledge the contribution of the WR community at large
for collaborating in the development of the technology, identifying
and reporting bugs and extensively testing each release of hardware,
gateware and software. Authorship of this paper does not imply a claim
to more importance inside the project, but rather a privilege to
report on the results of the collective effort of the many
contributors to the project.

%\begin{thebibliography}{9}   % Use for  1-9  references
\begin{thebibliography}{99} % Use for 10-99 references

\bibitem{white-rabbit-ref}
White Rabbit project website,\\\texttt{http://www.ohwr.org/projects/white-rabbit/wiki}

\bibitem{dmtd-ref}
D.A. Howe et al,. ``Properties of Signal Sources and Measurement
Methods,'' Proceedings of the 35$^{\textrm{th}}$ Annual Symposium on
Frequency Control, 1981.

\bibitem{wr-extension-ref}
M. Lipi\'{n}ski et al., ``White Rabbit: a PTP Application for Robust
Sub-nanosecond Synchronization,'' ISPCS 2011.

\bibitem{wishbone-ref}
Wishbone bus specification,\\\texttt{http://opencores.org/opencores,wishbone}

\bibitem{spec-ref}
Simple PCI Express Carrier website,\\\texttt{http://www.ohwr.org/projects/spec/wiki}

\bibitem{wrpc-ref}
WR PTP Core, \texttt{http://www.ohwr.org/projects/}\\
\texttt{wr-cores/wiki/Wrpc\_core}

\bibitem{etherbone-ref}
Etherbone Core
project,\\
\texttt{http://www.ohwr.org/projects/etherbone-core/wiki}

\bibitem{cernohl-ref}
CERN Open Hardware Licence,\\\texttt{http://www.ohwr.org/cernohl}

\bibitem{wr-users-ref}
WR users, \texttt{http://www.ohwr.org/projects/}\\
\texttt{white-rabbit/wiki/WRUsers}

\bibitem{wr-spec-ref}
White Rabbit Specification,\\
\texttt{http://www.ohwr.org/documents/160}

\bibitem{torture-ref}
WR torture report,\\
\texttt{http://www.ohwr.org/documents/190}

\bibitem{ppsi-ref}
PPSi,\\\texttt{http://www.ohwr.org/projects/ppsi/wiki}

\bibitem{ptp-wg-ref}
IEEE P1588 Working Group,\\
\texttt{https://ieee-sa.centraldesktop.com/1588public}

\bibitem{tsn-ref}
Time-Sensitive Networks Task Group,\\
\texttt{http://www.ieee802.org/1/pages/tsn.html}

\end{thebibliography}

\end{document}
