\section{Definition of reliability in a WRN}

A WRN, consisting of White Rabbit Switches (switches) connected by fiber 
or copper, is meant to transport information among White Rabbit Nodes (nodes). We distinguish 
two types of information distributed over the WRN: 
%(1) {\bf Timing} (frequency and International Atomic Time) and 
(1) {\it Timing} (frequency and Coordinated Universal Time) and 
(2) {\it Data} (the Ethernet traffic).
This translates into two types of services provided by the WRN which have their own requirements and
can be handled separately. The requirements are defined by GSI and CERN as the prospective 
users of WR to control their accelerators.


\subsection{Timing Distribution}

Timing is distributed in the WRN from a switch/node called Timing Master (TM) 
to all the other nodes/switches in the network. 
% The TM is usually connected 
% to the external source, such as Global Positioning System (GPS) receiver. 
All the devices in the 
WRN lock their frequency (syntonize) and adjust their local clocks (synchronize) to that of the TM. 
The deviation between the clock of the TM and that of any other node/switch is called {\bf accuracy}. 
A stable and continuous synchronization of all the nodes with an appropriate accuracy is the key 
requirement of the Timing Distribution in the WRN.

\subsection{Data Distribution}

The critical data distributed over the WRN is the one carrying sets of commands (events) which are 
organized into Control Messages (CM). The CMs are sent by a privileged node (Data Master, DM) in the 
payload of the Ethernet frame(s). Therefore, the Data Distribution in the WRN is broken into 
(1) {\it Control Data (CD)}  -- the Ethernet frames carrying CMs, critical, and 
(2) {\it Standard Data (SD)} -- the Ethernet frames which do not carry CMs, non-critical.
The reliability of the WRN depends on the successful delivery of the CD to all 
the designated nodes. The CMs are always broadcast within a VLAN
% \cite{bilbio:vlan}
, which can span 
the entire network. The worst-case upper bound of their delivery latency from the DM to any node in 
the network, regardless of it's location ({\bf maximum distance from the DM}), is required to be 
guaranteed by the network -- this is {\bf a determinism} requirement. 

\subsection{Reliability of the WRN}

The reliability of the WRN relies on the {\bf deterministic} delivery of the CD 
to all the designated nodes and their sufficiently {\bf accurate and stable synchronization}.  
This means that the WRN is considered non-functional if one or more of the following occur:
\begin{Itemize}
  \item A node is synchronized with insufficient accuracy.
  \item A designated node receives corrupted CD or no CD.
  \item The upper-bound delivery latency has been exceeded.
\end{Itemize}
% (1) A node is synchronized with insufficient accuracy;
% (2) A designated node receives corrupted CD or no CD;
% (3) The upper-bound delivery latency has been exceeded.
Unreliability is translated into the number of CMs considered lost (not delivered, delivered 
corrupted or in a non-deterministic way) in a given period of time. During this time,  
the synchronization must be always of the required quality. 
Quantitative requirements of the accelerator facilities are listed in Table~\ref{tab:requirements}.

\begin{table}[ht]
\caption{GSI's and CERN's requirements summary.} 
\centering
	\begin{tabular}{| l | c | c |}                        \hline
%\textbf{Requirement}& \multicolumn{2}{|c|}{\textbf{Value(s)}}     \\
\rowcolor{gray!35}{}
\textbf{Requirement}     & {\bf GSI}        & {\bf CERN}          \\ \hline
Max latency    		 & 100$\mu s$       & 1000$\mu s$         \\ \hline
CM failure rate          & $3.17*10^{-12}$  & $3.17*10^{-11}$     \\ \hline
CMs lost per year        & 1                & 1                   \\ \hline
$d_{max}$ from DM        & 2km              & 10km                \\ \hline
CM size 		 & 200-500 bytes    & 1200-5000 bytes     \\ \hline
Accuracy	  	 & probably 8ns     & 1$\mu s$ to  ~2ns   \\
%accuracy                 &                  & few nodes  ~2ns     \\
\hline

\end{tabular}
\label{tab:requirements}
\end{table}