\section{Determinism}

% The delivery latency of an Ethernet frame varies with cable length and the number of hops (switches) 
% it has to traverse to reach its destination, the traffic load on the way and 
% the assigned Class of Service (CoS, \cite{bilbio:vlan}). 
A carefully configured and properly used WRN offers deterministic Ethernet frame delivery 
thanks to the implementation of CoS and the fact that the delay introduced by the switch can be 
verified by analysis of {\bf publicly available source code} \cite{biblio:whiteRabbit}. 
Such analyses were performed to verify the worst-case upper bound 
delivery latency of a CM against the requirements listed in the Table~\ref{tab:requirements}. 
The results, presented in Table~\ref{tab:CMlatency} ({\it Store-and-forward} column), 
take into account the fact that a CM is encoded into 4 Ethernet frames (as required by the FEC 
and described in the next Section), it is sent with the highest priority (CoS) and it always 
traverses 3 hops.

\begin{table}[ht]
\caption{Control Message(CM) deliver latency estimations.} 
\centering

	\begin{tabular}{| c | c | c | c | c | c |}          \hline
\rowcolor{gray!35}{}
               & \multicolumn{4}{|>{\columncolor{gray!35}}c|}{\textbf{CM deliver latency}}                    \\ \cline{2-5}
\rowcolor{gray!35}{}
\textbf{CM size}& \multicolumn{2}{|>{\columncolor{gray!35}}c|}{\textbf{Store-and-forward}} 
                &\multicolumn{2}{|>{\columncolor{gray!35}}c|}{\textbf{Cut-through}}                           \\\cline{2-5}
\multicolumn{1}{|>{\columncolor{gray!35}}c|}{} &
\multicolumn{1}{|>{\columncolor{gray!35}}c|}{GSI} &
\multicolumn{1}{|>{\columncolor{gray!35}}c|}{CERN} &
\multicolumn{1}{|>{\columncolor{gray!35}}c|}{GSI} &
\multicolumn{1}{|>{\columncolor{gray!35}}c|}{CERN} \\ \hline
%		&    GSI           & CERN          &    GSI           & CERN          \\ \hline
%200 bytes      & ???$\mu s$       & ???$\mu s$    & ??$\mu s$        & ???$\mu s$    \\ \hline
500 bytes      & 221$\mu s$       & 283$\mu s$    & 76$\mu s$        & 118$\mu s$    \\ \hline
1500 bytes     & 285$\mu s$       & 325$\mu s$    & 102$\mu s$       & 142$\mu s$    \\ \hline
5000 bytes     & 324$\mu s$       & 364$\mu s$    & 162$\mu s$       & 202$\mu s$    \\ \hline
\end{tabular}
\label{tab:CMlatency}
\end{table}

The analysis revealed that GSI's requirements are not fulfilled: the upper-bound delivery latency
for the required size of CM and max distance of 2km is greater then 100$\mu s$. 

The solution to decrease delivery latency is targeted into the CD only and 
takes advantage of its characteristics (broadcast within a VLAN, sent by privileged node). 
We propose to break the highest priority of 
the CoS into two (unicast and broadcast) and use the highest priority broadcast Ethernet traffic only for 
the CD. Moreover, this particular traffic shall be forwarded using the cut-through method 
(unlike the store-and-forward method used normally in the switch) which can be effectively fast 
for the broadcast traffic with a single source (DM). The results, 
presented in Table~\ref{tab:CMlatency} ({\it Cut-through} column), show a significant improvement. 
The solution requires hardware-supported cut-through forwarding in the switch as described 
in \cite{biblio:robustness}. 

