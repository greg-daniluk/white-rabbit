\begin{abstract}

White Rabbit (WR) is a time-deterministic, low-latency Ethernet-based network which enables 
transparent, sub-ns accuracy timing distribution. It is being developed to replace 
the General Machine Timing (GMT) 
%\cite{biblio:GMT} 
system currently used at CERN and will become 
the foundation for the control system of the Facility for Antiproton and Ion Research (FAIR) 
at GSI. High reliability is an important issue in WR's design, 
since unavailability of the accelerator's 
control system will directly translate into expensive downtime of the machine. 
A typical WR network is required to lose not more than a single message per year. 
Due to WR's complexity, the translation of this real-world-requirement into 
a reliability-requirement constitutes an interesting issue on its own -- a WR network 
is considered functional only if it provides all its services to all its clients at any time. 
This paper defines reliability in WR and describes how it was addressed by dividing it into 
sub-domains: deterministic packet delivery, data 
%redundancy, 
resilience, 
topology redundancy and clock 
resilience. The studies show that the Mean Time Between Failure (MTBF) of the WR Network 
is the main factor affecting its reliability. Therefore, probability calculations for 
different topologies were performed using the "Fault Tree analysis" and analytic estimations. 
Results of the study show that the requirements of WR are demanding. Design changes might be needed 
and further in-depth studies required, e.g. Monte Carlo simulations. Therefore, a direction 
for further investigations is proposed.
\end{abstract}