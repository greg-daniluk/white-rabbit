\begin{abstract}
%\boldmath

  This article describes time distribution in a White Rabbit Network. 
  We start by presenting a short overview of the White~Rabbit
  project explaining its requirements to highlight the importance of
  the timing aspects of the system. We then introduce the
  technologies used to achieve high clock accuracy, stability and
  resilience in all the components of the network. In particular,
  the choice of the IEEE~1588-2008 (PTP) and Synchronous Ethernet 
  standards are explained. In order to accommodate 
  hardware-supported mechanisms to increase PTP synchronization accuracy, 
  we introduce the White Rabbit extension to PTP (WRPTP).
  The hardware used to support WRPTP is presented. 
  Measured results of WRPTP performance demonstrate sub-nanosecond accuracy over a 5km 
  fiber optic link with a precision below 10ps and 
  a reduced PTP-message exchange rate. Tests of the implementation show 
  full compatibility with existing PTP gear.
\end{abstract}
% IEEEtran.cls defaults to using nonbold math in the Abstract.
% This preserves the distinction between vectors and scalars. However,
% if the conference you are submitting to favors bold math in the abstract,
% then you can use LaTeX's standard command \boldmath at the very start
% of the abstract to achieve this. Many IEEE journals/conferences frown on
% math in the abstract anyway.

% no keywords
