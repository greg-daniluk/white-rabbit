\section{Synchronous Ethernet}


Synchronous Ethernet (SyncE, \cite{biblio:SynchE}) plays a very
important role in timing distribution over the WRN -- it is
responsible for clock syntonization. The reference clock is used to
encode the outgoing data stream.  The same clock is retrieved on the
other side of the physical link (by a so-called slave) using a
PLL. SyncE allows to use phase detector technologies to achieve
sub-nanosecond synchronization by increasing the precision of
timestamps far beyond the that allowed by the 125MHz clock resolution
(8ns).

The WR switch implementation of PLLs used to retrieve frequency from the
encoded data is designed to accommodate multiple sources of frequency (physical
links) with a single source being used (active) at a given time. A seamless
switching of the active source is one of the goals of the design. This enables
network topology redundancy. Additionally, the PLL uses a precise PTP link delay
measurement to correct the clock phase.

The WR clock recovery system uses Digital Dual Mixer Time Difference
(DDMTD) phase detection \cite{biblio:WRproject}. It consists of two
units: the Helper PLL (HPLL) and the Main PLL (MPLL), as depicted in
\figurename~\ref{}.

The MPLL uses offset frequency (\ref{equation:offsetFreq}, DMTD clock)
produced by the HPLL.
\begin{equation}
  \label{equation:offsetFreq}
     f_{offset}[ns] =  125[MHz] * \frac{2^N}{2^N+ \Delta}
\end{equation}

The DMTD clock produced by the VCXO is locked to the output of the
multiplexer of all the possible reference frequencies (uplinks or
external clock). The holdover unit incorporated into the HPLL
automatically switches the reference source in case of failure of the
currently used source.

The MPLL uses a DDMTD for each input channel (reference and feedback
clocks) to produce phase tags. The phase tags, corrected with the
phase shift obtained by WRPTP, are sent to the control
algorithm. Thus, the output reference clock produced by the MPLL is an
in-phase copy of the grandmaster clock. At any time, all the reference
channels are compared with the feedback channel and their frequency
and phase errors are ready for the dual PI controller. Such a solution
enables fast switching between different frequency sources. Although
the detection of reference channel failure is fast (3 symbols, X
time), to avoid instability of the produced reference clock, a delay
buffer can be implemented to accommodate the failure detection time.
