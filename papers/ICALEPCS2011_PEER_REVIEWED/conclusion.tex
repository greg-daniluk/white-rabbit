\section{Conclusions}


A WRN must be considered as an ordinary Ethernet network with extra optional built-in features 
which, when properly used, can make it robust and more reliable. This, however, comes at a price 
of topology restrictions and redundant elements (money). The reliability study described in this 
article and detailed in \cite{biblio:robustness} presents areas which need to be addressed to 
increase the reliability of a WRN. The development of WR is an on-going effort and some of the 
suggested solutions have been already properly investigated or developed (FEC, clock distribution) 
while the others need further verification (RSTP, cut-through forwarding). 
Suggested solutions enable to fulfill the requirements set by CERN and GSI. 
However the costs might trigger double-checking and re-justifying of at least two of them: 
upper-bound latency by GSI and the number of CMs lost per year.
The former requires additional development efforts to achieve the required 100$\mu s$. 
The latter requires a high level of network redundancy (triple or more) which is very costly. 
Since the network topology and its reliability calculations turned out to be the greater factor in 
the overall system reliability, it is necessary to perform more precise calculations and 
simulations to verify the rough estimations. This might include different techniques (e.g. Monte Carlo simulations) 
but also more real-life use cases (i.e. of the network layout suggested in 
\cite{biblio:CERNwrControlAndTiming}, which was not available at the time of described study). 
\modified{Especially, we need to take into account and include into calculations the fact that 
not all the nodes connected to the WRN are equally critical in real-life applications.}
