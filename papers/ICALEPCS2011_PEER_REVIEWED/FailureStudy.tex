\section{Failure Study}

One of the main possible reasons for WRN failure, which affects both Timing and Data Distribution, is 
a malfunction of its elements (switches or links). Since the distribution of information 
in the WRN is of one-to-all character (Data/Timing Master to all nodes), all the elements of the WRN are 
considered Single Points of Failure (SPoF)\cite{biblio:mtbf}. Malfunction of any SPoF 
results in failure of the entire system.
SPoFs can be eliminated by introducing redundancy of the system components. Due to its special features 
(distribution of frequency over physical layer) and strict requirements (determinism, low data loss), 
the number of possible redundant topologies of the WRN is restricted, as explained in the 
following sections. 

Imperfections of the physical medium as well as switching between redundant elements of the network 
(which takes time) can cause loss or corruption of data. The deterministic and \modified{mostly} broadcast character 
of the data distribution in the WRN enforces application of the Forward Error Correction (FEC) 
%\cite{biblio:coding} 
-- adding redundant information on transmission to enable recovery of lost or corrupted data 
on reception. This brings constant data overhead and the probability that the added redundancy is 
not sufficient to recover the data. However, it is the price to pay for ensuring low latency 
and determinism of data delivery in the WRN. 

The delivery latency of an Ethernet frame varies with cable length and the number of hops (switches) 
it has to traverse to reach its destination, the traffic load on the way and 
the assigned Class of Service (CoS). Therefore, to ensure the required determinism 
of the CD delivery, we need to make sure that there is no congestion of Ethernet frames 
carrying CMs. Moreover, the number of hops (the latency introduced by them) needs to be 
sufficiently small, which can be done by restricting the topology. 

The resilience of the Clock Distribution translates into continuous and stable 
synchronization of all the nodes and switches in the WRN (Table~\ref{tab:requirements}). Although, 
the network redundancy eliminates SPoFs, the switch-over between redundant elements might introduce 
instability and render the network unreliable despite the costly redundancy. 
Therefore, a seamless switch-over between redundant clock paths needs to be ensured. 
Another reason for the deterioration of the synchronization 
accuracy is the variation of external conditions (e.g. temperature) which needs to be compensated.

% In terms of the Data Distribution reliability, the topology redundancy can turn out to be 
% useless, if the switch-over between redundant elements causes more data to be lost then the 
% capabilities of FEC scheme.
% {\it [add here, change the rest]}
% In summary, we need investigate how to :
% \begin{Itemize}
%   \item  eliminate/decrease data loss due to :
%     \begin{Itemize}
%       \item physical medium imperfection,
%       \item switch over between redundant elements,
%       \item traffic congestion,
%     \end{Itemize}
%   \item eliminate synchronization instability due to:
%     \begin{Itemize}
%       \item switch over between redundant data paths,
%       \item external condition variations,
%       \item Ethernet frame loss (PTP),
%     \end{Itemize}
%   \item ensure required upper-bound delivery latency of Control Data.
% \end{Itemize}
