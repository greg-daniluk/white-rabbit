\section{Data Resilience}


\subsection{Forward Error Correction}
\label{sec:fec}

The objective of the FEC scheme is to decrease the loss rate of the CMs, preferably, 
to less then one per year. WR uses as a physical medium Fiber Optic and CAT-5. The number 
of received corrupted bits compared to the total number of received bits is called Bit Error Rate 
(BER). The value of BER  characterizes a physical medium and can be used to characterize the entire 
switched network.
%  if the following factors are taken into account: 
% (1) {\it type of cabling} (fiber/twisted pair),
% (2) {\it logic topology},
% (3) {\it network address} (broadcast/unicast).
A WRN can be seen as a Packet Erasure Channel (PEC) or as a Binary Erasure Channel (BEC) depending 
on the effect of a bit error on the frame. If the frame is lost (e.g. dropped by the switch due to 
a corrupted header or lost during switch-over between redundant components), the WRN is a PEC. 
If the bit error happens in the link between a switch and node, a corrupted frame 
\modified{can be used (optional)}
%is used 
to attempt frame recovery. In such case, the channel is called BEC. Each type of channel requires 
a different FEC solution. Therefore two concatenated FECs are used in WR. 
Reed-Solomon (R-S) %\cite{biblio:r-s} 
%\cite{biblio:coding} 
coding is used for the PEC and 
allows to encode k original-frames into n encoded-frames ($n>k$). 
Reception of any k encoded-frames can be used to decode the original frames. 
\modified{Hamming coding with additional parity (SEC-DED)} 
%Hamming coding 
is used for the BEC and allows to detect up to two simultaneous bit errors and 
correct a single error.
These two schemes (R-S and Hamming) are combined to encode each CM -- it is 
split into two and encoded using R-S into four messages (two original and two 
with redundant data). Each of the four messages is then encoded using Hamming. Such encoded 
messages are sent in a burst of 4 Ethernet frames. Reception of any two of these frames enables 
to decode the original Control Messages. 
A systematic analysis, using the BER characteristic of the WRN, proves that the presented FEC scheme 
guarantees less than single CM lost per year due to physical medium 
imperfection, as can be seen from Table~\ref{tab:gsi_cern_fec}. 

\begin{table}[ht]
 	\begin{center}
\caption{GSI and CERN FEC characteristics.}
\begin{tabular}{|p{4cm}|c|c|} \hline
%	\cline{2-3}
%	&  \multicolumn{2}{|c|}{Use Case} \\ \cline{2-3}
\rowcolor{gray!35}{}
{\bf Parameter}	&  {\bf GSI} & {\bf CERN} \\ \hline
	\multicolumn{1}{|p{4cm}|}{Control Message length} & 500 bytes & 1500 bytes     \\ \hline
	\multicolumn{1}{|p{4cm}|}{Control Message per year} & $3.145 10^{11} $ &$  3.145 10^{8} $ \\ \hline
	\multicolumn{1}{|p{4cm}|}{Max Bit Correct.} & 1 & 1  \\ \hline
%	\multicolumn{1}{|p{4cm}|}{Parity-Check Bits} & 13    &  13   \\ \hline
%	\multicolumn{1}{|p{4cm}|}{PEC Code Overhead} & 3  & 2 \\ \hline
%	\multicolumn{1}{|p{4cm}|}{Payload Length} & 400 b  & 800b \\ \hline
	\multicolumn{1}{|p{4cm}|}{Payload Length} & \modified{294 bytes}  & \modified{854 bytes} \\ \hline
	\multicolumn{1}{|p{4cm}|}{Num Encoded Frames} & 4  & 4 \\ \hline
	\multicolumn{1}{|p{4cm}|}{Needed Frames to Receiver} & 2 & 2 \\ \hline
	\multicolumn{1}{|p{4cm}|}{Probability of Loosing a CM} & $10^{-14}$ & $10^{-13}$\\ \hline
	\end{tabular}   
	
	\label{tab:gsi_cern_fec}
	\end{center}
\end{table}

\subsection{Rapid Spanning Tree Protocol (RSTP)}

In an Ethernet network with redundant topology, the problem of loops (causing "broadcast storms") 
is handled by the Rapid Spanning Tree Protocol (RSTP)
% \cite{biblio:IEEE8021D}
. It creates a loop-free 
logical topology by blocking appropriate ports in switches, and unblocks them in case of topology 
break (due to element failure).

The functionality provided by the RSTP is essential for the WRN. However, the convergence speed 
provided by the standard implementation of the RSTP (milliseconds 
%\cite{biblio:RSTPperf} 
at best) 
would cause many CMs to be lost during the process. This is not acceptable, we need 
a solution which is fast enough to prevent loosing the CMs at all. Since we know the 
size-range of the CMs (Table~\ref{tab:requirements}) and how they are FEC-encoded into Ethernet frames,
we can calculate the maximum value of the convergence time: 3$\mu s$. This time is smaller than 
the duration of transmitting a single frame with FEC-encoded CM -- this ensures that no more than 
two frames with FEC-encoded CM are lost, thus the CM can be recovered.

In order to achieve a convergence time of 3$\mu s$, the switch-over between active 
and backup connections needs to be performed in the hardware as soon as the link-down is detected. 
It can only be done if the alternative topology is known in advance. The knowledge of alternative 
topology is translated into an RSTP-assignment of alternative and backup roles of switch ports, 
i.e at least one port with alternative role must be identified in every switch 
(except the topology-root switch).    
%\modified{, i.e at least one port with each of these roles must be identified in every switch}.
%
%If at least one port of a switch is assigned an alternative role, it means that 
%the RSTP algorithm establishes more than one path to the topology-root switch and therefore 
%the alternative topology is know in advance. 
%Such ports are identified when the RSTP algorithm establishes more than one path to the 
%topology-root switch and all paths can be used simultaneously, 
%
If we ensure, by restricting the topology, that RSTP identifies the alternative links, 
we can use its data to feed the hardware, consequently achieving the required convergence time 
and staying standard-compatible: 
the hardware switch-over is just a faster RSTP-driven convergence. The required topology 
restrictions, described in \cite{biblio:robustness}, greatly overlap with these imposed by 
the Time Distribution. 

 

