\section{Introduction}

The WR project is a multi-laboratory, 
multi-company, international effort to create a universal fieldbus for control and timing systems 
to be used at CERN, GSI and possibly other such facilities. The rationale behind WR, 
the choice of the technologies and technical details of its functioning have been already 
described in a number of papers \cite{biblio:WRproject}, \cite{biblio:TomekMSc}, 
\cite{biblio:WRPTP}. 
%, \cite{biblio:ISPCS2011}. 
The resilience and robustness is one of the key features of any fieldbus. 
This article presents a study on the reliability of a White Rabbit Network (WRN) 
assuming a basic knowledge about WR. 

Reliability is defined as the ability of a system to provide its services to clients under both 
routine and abnormal circumstances. It can be estimated by calculating the probability of 
the system's failure ($P_f$). 
% \begin{equation}
%   \label{eq:reliability}
%   R =1 - P_f
% \end{equation}
The lesser the probability of WRN failure, the higher its reliability. Thus, in this article we 
identify critical services of a WRN based on the study of WR's requirements. 
Then, we analyze each critical service to identify possible 
reasons for their failure and propose targeted counter-measures to increase reliability. 
Finally, their impact on the overall system reliability is studied to 
identify the highest contributor and the focus for the further studies.

