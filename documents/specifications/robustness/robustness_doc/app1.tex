\chapter{Appendix: Reliability measure (Mean Time Between Failures)}
\label{appA}



The measurement of reliability by the number of Control Messages lost per year, 
which has been mentioned in the requirements, is neither standard nor practical.
Therefore, in this document, we estimate reliability of White Rabbit Network 
using method which is readily and commonly applied to Large-Scale LANs. We use 
Mean Time Between Failures of a single network component (be it WR Switch, 
fibre/copper link, WR Node) to estimate reliability (i.e. MTBF and failure
probability) of entire White Rabbit Network as described in
\cite{DesigningLSLANs}. 

Mean Time Between Failures (MTBF) represents a statistical likelihood that half 
of the number of devices, represented by a given MTBF factor, will not function 
properly after the period given by MTBF. MTBF does not give the functional 
relationship between time and number of failures. However, the estimation that 
the function is linear is assumed to be sufficient (see \cite{DesigningLSLANs}, 
page 36).

For network MTBF derivation, the probability of failure related to the number of
failures of N devices per unit time is interesting:
  \begin{equation}
     \label{eq:MTBFprob}
	P_= \frac{N}{2*MTBF}  
  \end{equation}

However, to use in probability calculations, a net value is required.
Therefore, the below equation is used. It is a probability of single device
failing in a single day, where M denotes MTBF 
per-day:

  \begin{equation}
      \label{eq:MTBFprobNetto}
	P_= \frac{1 }{2*M}  
  \end{equation}

Examples of common values of MTBF of network components and corresponding 
probability of their failure per-day are presented in 
Table~\ref{tab:MTBFtable}.


\begin{table}[ht]
\caption{Example MTBFs and probabilities of network units (src: 
	\cite{DesigningLSLANs})} 
\centering
	\begin{tabular}{| l |  c | c |}          \hline
\textbf{Component} & \textbf{MTBF [hours]} & \textbf{Probability [$\%$]}\\ 
                   &                       &                           \\ \hline
Fiber connection   &       1000 000         & 0.0012                    \\
\hline
Router             &       200 000         & 0.0060                    \\ \hline
\end{tabular}
\label{tab:MTBFtable}
\end{table}

In order to calculate reliability of entire network, the meaning of network 
failure needs to be defined. In case of White Rabbit Network, it is critical 
that all the White Rabbit Nodes connected to the network receive Control 
Messages. In other words, failure for White Rabbit Network is a failure of any 
number of its components which prevents any WR Node from receiving Control 
Messages. If single component causes network failure, such component is called 
a single point of failure (SPoF).

The probability of entire network failure is calculated by adding probabilities
of all the components' simultaneous failures combinations which cause failure of
the entire network. 
Using equation~\ref{eq:MTBFprobNetto}, MTBF of entire network can be calculated 
out of network failure probability.
