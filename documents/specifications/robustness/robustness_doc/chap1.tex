\chapter{Introduction}
\label{introduction}

\section{What is White Rabbit?} 

White Rabbit is intended to be the next-generation deterministic network based
on synchronous Ethernet, allowing for low-latency deterministic packet routing
and transparent, high precision timing transmission. The network consists of
White Rabbit Nodes, White Rabbit Switches and supports integration of nodes
and/or switches that are not White Rabbit, however with restrictions.


The resilience and robustness is one of the key features of any
fieldbus, specially, in safety Ethernet-based fieldbuses for critical systems
like White Rabbit. The reliability of the WR falls on the deterministic delivery
of Ethernet frames through a switching network and the synchronization of the
network devices. In order to provide a service with low error, it is necessary
to propose methods and techniques to overcome problems caused by the
imperfection of the physical medium, dropped packets in WR switches and
breakdowns of the network devices.


A White Rabbit Network, consisting of White Rabbit Switches (WRS) connected
by fibre or copper, is meant to transport information between White Rabbit
Nodes (WRN). In this document three types of information transported over White
Rabbit Network are distinguished:

\begin{itemize}
  \item  Timing Information - includes frequency and Coordinated Universal
  Time (UTC), it is sent from Timing Master to White Rabbit Switches and Nodes.
  \item  Control Data - includes \ControlMessage s (\CM), it is broadcast from
Data Master to  White Rabbit Nodes. 
  \item  Standard Data - all other Ethernet traffic sent between nodes and
  switches.
\end{itemize}
Timing Information and Control Data are considered to be critical. 
The types of information are closely related to their source. A Timing Master is
a White Rabbit Switch or Node which is connected to GPS receiver. A Data Master
is a White Rabbit Node which is responsible for \ControlMessage\   distribution.


The main component of White Rabbit Network is a White Rabbit Switch.
It is a Layer~2 network bridge which supports Synchronous Ethernet
(SyncE) \cite{SynchE} and implements White Rabbit Protocol (WRPTP) \cite{WRPTP}
which is an extension of PTP standard (IEEE1588 \cite{IEEE1588}). WRPTP, along
with SyncE, enables to distribute common notion of time and frequency over the
entire White Rabbit Network with sub-nanosecond precision. 

\section{WR Network Requirements} 

The requirements for the WR have been defined by GSI and CERN, since the
Control Sytems of their accelerator are going to be based on WR. A Control
System for accelerator facility requires high reliability, high precision and
determinism. 

Unreliability is translated into the number of \ControlMessage s not delivered
to one or more designated nodes within one year. 

In terms of reliability, it is expected to lose along the way from WR Data
Master to receiving WR Nodes only one \ControlMessage\ per year. 

Determinism in this chapter is understood as delivery of \ControlMessage\ to all
designated nodes within the required time regardless of the node's location in
the network. If the \ControlMessage\ delivery time is exceeded, the message is
considered undelivered (lost).

At GSI, White Rabbit is foreseen to control new FAIR accelerator complex. The
requirements for the new FAIR machines \cite{FAIRtiming} state that the Control
Message should be delivered from Data Master Node to all receiving devices
(Nodes) within 100$\mu s$ given the maximum distance between Data Master and a
Node of 2km \cite{FAIRtimingSystem}.

At CERN, the required time to delivered the message is 1ms but the distance 
between Data Master and a Node is substantially larger (max of 10km)
\cite{CERNtiming}.

Knowing how often a \ControlMessage\ is sent (every 100$\mu s$ at GSI and 
1000$\mu s$ at CERN), we can calculate the maximum acceptable failure rate
of WR Network ($\lambda_{WRN_{max}}$)

\begin{equation}
 \label{eq:failureProbability}
	\lambda_{WRN_{max}} = \frac{number\_of\_\ControlMessage
s\_lost\_per\_year}{number\_of\_\ControlMessage s\_sent\_per\_year}
\end{equation}


The requirements concerning \ControlMessage\ size differ between FAIR and CERN
as well. While in GSI facility a \ControlMessage\ of 1500 bytes (maximum
Ethernet Frame payload) is more than sufficient, for CERN facility it is an
acceptable value, but the expectations are higher, e.g.: 4800bytes (see
Appendix~\ref{appB}). Table~\ref{tab:requirements} summarizes the GSI's and
CERN's requirements.

\begin{table}[ht]
\caption{GSI's and CERN's requirements summary.} 
\centering
	\begin{tabular}{| c | c | c |}          \hline
\textbf{Requirement name}& \multicolumn{2}{|c|}{\textbf{Value(s)}}  \\
                         & GSI              & CERN          \\ \hline
\GranularityWindow       & 100$\mu s$       & 1000$\mu s$   \\ \hline
max Failure rate ($\lambda_{WRN_{max}}$) & $3.170979198*10^{-12}$ &
$3.170979198*10^{-11}$  \\ \hline
%min Reliability ($R_{WRN}$)& 0.999 999 999 997
% & 0.999 999 999 968\\ \hline
Maximum Link Length      & 2km              & 10km          \\ \hline
\ControlMessage Size     & 300-1500 bytes   & 1200 - 5000 bytes     \\ \hline
Synchronization accuracy & probably 8ns   & most nodes 1$\mu s$ \\
                         &                & few nodes  ~2ns \\
\hline

\end{tabular}
\label{tab:requirements}
\end{table}

\section{The Goals of This Document} 

This document introduces methods and techniques which increase 
reliability, robustness and determinism of White Rabbit Network so that the
requirements listed in the previous chapter are met. It also introduces
techniques to guard the safety of the network, monitor and 
diagnose. However, the presented solutions are considered optional. Their usage
should be considered on individual bases and according to actual needs of 
a particular use case.



Determinism, as understood in this document, can be achieved by a precise
knowledge of maximum delays introduced by each component of the network and 
optimisation of the delay to meet the requirements. 

\textbf{Chapter 2} discusses determinism of White Rabbit Network. In particular
it focuses on the \ControlMessage\ maximum delivery time from the Data Master
to WR Nodes. 

Unreliability of the system is caused by physical imperfection of network's
components. In particular, by data corruption on the physical links and failure
of network components. Reliability of data distribution in a network can be
increased by redundancy of data (special encoding) and components
(switches, links). In terms of redundancy, two types of information
distributed over White Rabbit Network are considered separately: Timing
Information and Data (Control and Standard). 

Timing Information is conveyed from a Timing Master to all nodes. Data is
sent from any node (Data Master, in case of Control Data) to one or many nodes.
Timing Information and Data are distributed along so called Paths. A Path is
understood as the cables and switches by which information traverse from the
sender to the receiver of the information. Consequently, there are two types of
paths:
\begin{itemize}
  \item Clock Path - path from Timing Master to the nodes.
  \item Data Path - path from sending node to receiving node(s).
\end{itemize}
A layout of all possible Paths in a network is called topology. Therefore we
define in White Rabbit Network:

\begin{itemize}
  \item Clock Path Topology - Layout of all possible Clock Paths in the
network.
  \item Data Path Topology - Layout of all possible Data Paths in the
network.
\end{itemize}

One of the unique features of White Rabbit Network is a precise synchronization
and syntonization of all the White Rabbit network components. Failure to deliver
this information to any White Rabbit component, or deterioration of its quality
(instability) might have sever consequences and render the network unreliable.
\textbf{Chapter 3} describes how the quality of Timing Information is
guarded, its reliability increased by introducing redundancy of Clock Path and
its stability ensured during switchover between redundant Clock Paths.

\textbf{Chapter 4} describes techniques to achieved required reliability of
Control Data delivery and substantially increase reliability of Standard Data.
This is done by introducing redundancy of network components (topology) and
redundancy of data (Forward Error Correction, EFC). Since, congestion of network
traffic is undesired and dangerous for reliable message delivery, this chapter
deals with flow control and congestion problems as well. 

\textbf{Chapter 5} focuses on techniques which allows for WR Network
monitoring as well as fast and efficient diagnostics of the network.

