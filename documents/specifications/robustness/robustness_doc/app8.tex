\chapter{Appendix:Flow Monitor} 
\label{appSFlow}

\section{sFlow}

sFlow is a multi-vendor sampling technology embedded within switches and
routers. It provides the ability to continuously monitor application level
traffic flows at wire speed on all interfaces
simultaneously~\ref{tab:sflow_info}.


sFlow consists of:

\begin{itemize}
	\item sFlow Agent 
	\item sFlow Collector
\end{itemize}

The sFlow Agent is a software process that runs in the White Rabbit Switches.
 It combines interface Counters and Flow Samples into sFlow datagrams that are
sent across the network to an sFlow Collector. The Counters and Flow Samples
will be implemented in hardware in order to increase the
processing of the data sampling. Flow samples are defined based on a sampling
rate, an average of 1 out of N packets is randomly sampled. This type of
sampling provides  quantifiable accuracy. A polling interval defines how often
the network device sends interface counters. 

The sFlow Agent packages the data into sFlow Datagrams that are sent on the
network. The sFlow Collector receives the data from the Flow generators, stores
the information and provides reports and analysis. 

\subsubsection{Configuration}

Every switch capable of sFlow must configure and enable:

\begin{itemize}
	\item local agent
	\item sFlow Colector address 
	\item ports to monitor
\end{itemize}

In order to acquire a reliable network information in a WR network:

\begin{itemize}
	\item the statistics shall be collected every ?? (sec,msec..)
	\item a sample is taken per port every ?? (sec,msec...)
	\item ?? samples per port shall be sent to the CPU

\end{itemize}


\section{Requirements of a Flow Monitor}

General requirements:
\begin{itemize}
	\item Network-wide view of usage and active switches. 
	\item Measuring network traffic, collecting, storing, and analysing
traffic data.
	\item Monitor links without impacting the performance of the switches
without adding significant network load.
	\item Industrial Standard
\end{itemize}

\noindent The Flow Monitor shall: 

\begin{itemize}
	\item Measure the volume and rate of the traffic by QoS level.
	\item Measure the availability of the network and devices.
	\item Measure the response time that a device takes to react to a given
input.
	\item Measure the throughput of the over the links.
	\item Measure the latency and jitter of the network.	
	\item Identify grouping of traffic by logics groups (Master, Node,
Switch)
	\item Identify grouping of traffic by protocols.
	\item Define filters and exceptions associated with alarms and
notification.
\end{itemize} 


\noindent The measurements shall be carried out either between network devices,

\noindent Per-Link Measurements, and monitor:

		\begin{itemize}	
			\item number of packet
			\item bytes
			\item packet discarded on an interface
			\item flow or burst of packets
			\item packets per flow
		\end{itemize}

\noindent or End-to-End Measurements:
		\begin{itemize}	
			\item path delay  
			\item ....
			\item ....
		\end{itemize} 

\noindent The combination of both measurements provides a global picture of the
network.


\vspace{10 mm}

\noindent The monitoring shall performance:

\begin{itemize}
	\item Active Measurement, injection of network traffic and study the
reaction to the traffic
	\item Passive Measurement,  Monitor of the traffic for measurement.
\end{itemize}

\vspace{10 mm}

\noindent Performance:

\begin{itemize}

	\item Reaction Time ... 
	\item Sampling...
	
\end{itemize}


\section{State of the Art of Flow Controller}

Currently there are three main choices for traffic monitoring:

\begin{itemize}

	\item RMON, IETD standard.
	\item NetFlow, Cisco Systems.
	\item sFlow, Industry standard
\end{itemize}

In a nutshell, all them offers the similar features and provides the same
information, thus the selection criteria is based on the usage of resources by
the Agent in the switches and the collector of information.

\begin{table}[ht]
\begin{center}
    \begin{tabular}{ | c | c | c | c | c | c | c |}
\hline
Flow Controllers & CPU & Memory & Bandwidth & RT Statistics & Implementation \\
\hline
RMON & high &  very high 8-32 MB & bursty & supported & sw \\ \hline
NetFlow & high & high 4-8 MB & high bursty & not & sw  \\ \hline
sFlow & very low & very low akB & low smooth & supported & sw/hw \\ \hline
    \end{tabular}
\end{center}
\caption{Comparison Flow Control}
\end{table}

As the Table~\ref{tab:flow_controlers} shows that sFlow requires less resources
either in the Agent, which is placed in the switch, or the Collector. As well
the usage of bandwidth is more conservative since the gathered information every
short periods of time, conversely to the others controllers. It seems that
sFlows becomes a good choice for White Rabbit. Besides sFlows allows the
implementation of part of Agent in hardware, providing wire-speed to the
sampling of frames. In addition the license scheme of sFlow's  allows White
Rabbit project modify and publish our own version.

