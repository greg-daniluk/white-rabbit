%\chapter{Flow and Congestion Control}
\section{Flow and Congestion Control}
\label{chap:flow_congestion}

As a part of reliable network implementation, Flow Control and Congestion
Control are responsible for ensuring that data is transmitted at a rate coherent
with the capacities of both receiver and switches. Flow Control aims at
preventing congestion in the network while the Congestion Control provides the
mechanism to overcome the congestion.

%%%%%%%%%%%%%%%%%%%%%%%%%%%%%%%%%%%%%%%%%%%%%%%%%%%%%%%%%%%%%%%%%%%%%%%%%%%%%%%%

\subsection{Flow Control}
It provides a mechanism for the receiver to control the transmission, so that
the receiving node is not overwhelmed with data from transmitting node. 
\cite{atm_traffic}. 

\vspace{10 mm}

\subsubsection{White Rabbit Flow Control}

Since in WR we distinguish two types of traffic (\HP\ and \SP) and the most
important traffic falls on the \HP\ which is treated in a special way, two
different levels of flow control are needed. The configuration of flow control
is gathered in Flow Control Policy.

In a White Rabbit network, the \HP\ will flow from the Data Master Node to all
White Rabbit Nodes. DMN is the only node that can send \HighPriority\ 
frames \footnote{Recommended configuration}. 

There are two situations regaring the flow of the \HP\ Traffic that could point
out a malfunction or wrong configuration of a Node and cause congestion in the
network:
\begin{itemize}
    \item Data Master sents more frames that it should send,
    \item Non-Data Master Node sends \HP\ frames.
\end{itemize}
Therefore a simple but effective Flow Control mechanism is proposed for this
situation. In the first case, the Data Master shall be notified so that it can
perform appripriate action to resume propre \HP\ packages sending rate.
Destructive consequences on the \HP\ Traffic of the second problem are prevented
by blocking all ports connected to non-Data Master Nodes for \HP\ Traffic. The
Appendix ~\ref{flow_control} presents a proposal for the Flow Control of the
\HP\ traffic.
 
For the \SP Traffic, the Ethernet Flow Control described in the IEEE 802.3
\cite{IEEE8023} standard is used. The downside of this scheme is the lack of CoS
criteria, all the priorities of CoS are treated equally.  The authors of
this document will follow the development of IEEE 802.1qbb \cite{IEEE8021Qbb}
specification where the different level of the CoS are taken into account and
see the suitability of the standard in WR. 

\vspace{10 mm}

%%%%%%%%%%%%%%%%%%%%%%%%%%%%%%%%%%%%%%%%%%%%%%%%%%%%%%%%%%%%%%%%%%%%%%%%%%%%%%%%
\subsection{Congestion Control}

Congestion control is responsible for the control and regulation of the traffic 
into WR Network. The goal is to avoid saturating or overloading switches
in the network. The incoming traffic in a switch $\lambda_in$ should be equal to
the outgoing traffic $\lambda_out$ . When $\lambda_out \leq \lambda_in$, there
is a situation of congestion and the symptom are:
\begin{itemize}
	\item Lost packets, buffer overflow,
	\item Long delays, queueing in buffers
\end{itemize}

\noindent and it causes:

\begin{itemize}
	\item Increased delay,
	\item Packet loss.
\end{itemize}

%\subsubsection{WR Explicit Congestion Signalling for \HP}
\paragraph{WR Explicit Congestion Signalling for \HP}

Among the different schemes for Congestion Control, the Explicit Congestion 
Signalling is the scheme that fulfils the responsiveness and reliability
that \HP requires since the scheme avoids the congestion, consequently the loss
of frames due to buffer overflow.

The aim of an explicit signalling is to stop a device of sending traffic to
avoid the congestion. 


