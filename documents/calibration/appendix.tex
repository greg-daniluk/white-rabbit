\section{Mathematical proofs}

\subsection{Reference fiber latency}
\label{subsec:app_filat}
The same pair of WR Devices is used for all three connections in this procedure.
That is why, when considering round-trip delay after subtracting the bitslide
values, the transmission and reception delays of both devices are summed
together and remain constant for fiber $f_1$, $f_2$, $f_1 + f_2$:
\begin{equation}
	\Delta = \Delta_{TXM} + \Delta_{RXM} + \Delta_{TXS} + \Delta_{RXS}
\end{equation}

When two fibers $f_1$, $f_2$ are joined together the fiber latency for this
connection will be the sum of $\delta_1$ and $\delta_2$. After eliminating the 
bitslide value, the remaining part of a round trip delay consists of the 
following elements:
\begin{equation}
	delay_{MM1}' = \Delta + \delta_1\\
\end{equation}
\begin{equation}
	delay_{MM2}' = \Delta + \delta_2\\
\end{equation}
\begin{equation}
	delay_{MM3}' = \Delta + \delta_1 + \delta_2
\end{equation}

This equation system has three unknowns and after solving gives the formulas for
the round-trip fiber latencies:
\begin{align}
	\delta_1 = delay_{MM3}' - delay_{MM2}'\\
	\delta_2 = delay_{MM3}' - delay_{MM1}'
\end{align}

\subsection{Fiber asymmetry}
\label{subsec:app_fiasym}

In this step of the calibration procedure two WR connections with the same pair
of devices are established. For each of them the offset between the WR Slave and
WR Master is calculated by the WR PTP software as:
\begin{equation}
	offset_{MS} = t_1 - t_2 + delay_{MS}
\end{equation}

\noindent where $delay_{MS}$ is an estimated one-way link delay. When $\alpha$ is
initially equal to 0, $delay_{MS}$ is estimated as half of the round-trip
delay, which results in a distorted offset between the two devices:
\begin{equation}
	offset_{MS}' = t_1 - t_2 + \frac{1}{2} delay_{MM}
\end{equation}

\noindent Then $skew_{PPS}$ measured with an oscilloscope is equal to the
uncompensated link asymmetry (the sum of the fiber asymmetry and hardware
asymmetry):
\begin{align}
  \label{equ:app_fiasym:skew1}
  skew_{PPS1} &= offset_{MS1} - offset'_{MS1} = delay_{MS1} - \frac{1}{2} delay_{MM1}\\
  \label{equ:app_fiasym:skew2}
  skew_{PPS2} &= offset_{MS2} - offset'_{MS2} = delay_{MS2} - \frac{1}{2} delay_{MM2}
\end{align}

\noindent We also know what factors build up the round trip delays and one-way
delays for both connections. Please notice since the same pair of the devices is
used in both cases, fixed hardware delays stay the same:
\begin{align}
  delay_{MM1} &= \Delta + \delta_1\\
  delay_{MM2} &= \Delta + \delta_2\\
  delay_{MS1} &= \Delta_{TXM} + \Delta_{RXS} + \delta_{MS1}\\
  delay_{MS2} &= \Delta_{TXM} + \Delta_{RXS} + \delta_{MS2}\\
     \delta_1 &= \delta_{MS1} + \delta_{SM1}\\
     \delta_2 &= \delta_{MS2} + \delta_{SM2}
\end{align}

\noindent Using the formulas above, equations \ref{equ:app_fiasym:skew1} and
\ref{equ:app_fiasym:skew2} can be expanded:
\begin{align}
  skew_{PPS1} &= \Delta_{TXM} + \Delta_{RXS} + \delta_{MS1} - \frac{1}{2}\Delta
  - \frac{1}{2} \delta_1\\
  skew_{PPS2} &= \Delta_{TXM} + \Delta_{RXS} + \delta_{MS2} - \frac{1}{2}\Delta
  - \frac{1}{2} \delta_2
\end{align}

\noindent Subtracting the two skew measurements eliminates any asymmetry due to
fixed hardware delays:
\begin{align}
  \label{equ:app_fiasym:skew_pps}
  skew_{PPS} &= skew_{PPS2} - skew_{PPS1}\\
             &= \Delta_{TXM} + \Delta_{RXS} -
  \Delta_{TXM} - \Delta_{RXS} + \delta_{MS2} - \delta_{MS1} - \frac{1}{2}\Delta + \frac{1}{2}\Delta -
  \frac{1}{2} \delta_2 + \frac{1}{2} \delta_1 \nonumber\\
  &= \delta_{MS2} - \delta_{MS1} - \frac{1}{2}\delta_2 + \frac{1}{2}\delta_1
\end{align}

\noindent However, if fiber $f_1$ is just a few meters long, then its asymmetry is
negligible. That means its one-way Master-to-Slave latency equals half of the
total fiber latency:
\begin{equation}
  \delta_{MS1} = \frac{1}{2} \delta_1
\end{equation}

\noindent This results in a simplified formula describing $skew_{PPS}$:
\begin{align}
  skew_{PPS} &= \delta_{MS2} - \frac{1}{2}\delta_2 = \delta_{MS2} -
  \frac{1}{2}\delta_{MS2} - \frac{1}{2}\delta_{SM2} \nonumber\\
  \label{equ:app_fiasym:final_skew}
  &= \frac{1}{2}(\delta_{MS2} - \delta_{SM2})
\end{align}

\noindent Having in mind that $\alpha = \frac{\delta_{MS} - \delta_{SM}}{\delta_{SM}}$,
using the already known value of the $f_2$ round-trip latency $\delta_2$ and
equations \ref{equ:app_fiasym:skew_pps}, \ref{equ:app_fiasym:final_skew} we get
the expression for $\alpha$ used in the calibration procedure:

\begin{equation}
	\alpha = \frac{2(skew_{PPS2} - skew_{PPS1})}{\frac{1}{2}\delta_2 -
	(skew_{PPS2} - skew_{PPS1})}
\end{equation}

\subsection{WR Device calibration}
\label{subsec:apx:devices}

After the WR PTP daemon on a Slave device is synchronized to Master, the
$skew_{PPS}$ observed on an oscilloscope can be treated as an error of
a clock correction on the Slave side:
\begin{equation}
	\label{equ:devices:corrs}
	corr = corr_{ideal} - skew_{PPS}
\end{equation}

The correction value that should be applied to the Slave clock by the daemon
($corr_{ideal}$) is calculated based on timestamps and a $delay_{MS}$ estimation:
\begin{equation}
	corr_{ideal} = t_1 - t_2 + delay_{MS_{ideal}}
\end{equation}
The one-way delay is the sum of the fiber latency, Master transmission delay and
Slave reception delay:
\begin{equation}
	\label{equ:devices:ideal_delay}
	delay_{MS_{ideal}} = \frac{1+\alpha}{2+\alpha}(delay_{MM} - \Delta) + 
	\Delta_{TXM} + \Delta_{RXS}
\end{equation}

However, the Slave reception delay used by the daemon is the result of the first
4 steps of the procedure in \ref{subsec:devices} ($\frac{1}{2}\Delta_S$). That means,
it has to be corrected by an asymmetry coefficient $\beta$ to get the right 
value that produces $corr_{ideal}$ above:
\begin{equation}
	\label{equ:devices:delta_rxs}
	\Delta_{RXS} = \frac{1}{2}\Delta_S + \beta
\end{equation}

The round-trip delay value and the sum of hardware delays are fixed,
which means the same asymmetry factor has to be subtracted from the Slave
transmission delay to preserve those sums:
\begin{equation}
	\label{equ:devices:delta_txs}
	\Delta_{TXS} = \frac{1}{2}\Delta_S - \beta
\end{equation}

\noindent Taking it back to equation \ref{equ:devices:ideal_delay} we get:
\begin{equation}
	delay_{MS_{ideal}} = \frac{1+\alpha}{2+\alpha}(delay_{MM} - \Delta) + 
	\Delta_{TXM} + \frac{1}{2}\Delta_S + \beta
\end{equation}

However, the Master to Slave delay calculated by the daemon using the values without
the asymmetry taken into account is:
\begin{equation}
	delay_{MS} = \frac{1+\alpha}{2+\alpha}(delay_{MM} - \Delta) + 
	\Delta_{TXM} + \frac{1}{2}\Delta_S
\end{equation}

So the correction value for the reception asymmetry is also the difference
between the $delay_{MS}$ estimations:
\begin{equation}
	delay_{MS_{ideal}} = delay_{MS} + \beta
\end{equation}

\noindent Putting this back into the equation for $corr_{ideal}$:
\begin{equation}
	corr_{ideal} = t_1 - t_2 + delay_{MS} + \beta
\end{equation}

\noindent Please remember though, $t_1 - t_2 + delay_{MS}$ is in fact the correction
value ($corr$) derived from the coarse (without asymmetry) Slave delays:
\begin{equation}
	corr_{ideal} = corr + \beta
\end{equation}

Comparing the equation above with \ref{equ:devices:corrs}:
\begin{equation}
	\beta = skew_{PPS}
\end{equation}

That means, the difference between 1-PPS signals observed on the oscilloscope
has to be used as the correction factor for the coarse delays of the Slave
device.\\

The asymmetry of each calibrated Tx/Rx delay is set to compensate also the
asymmetry of the WR Calibrator. Equations \ref{equ:devices:delta_rxs} and
\ref{equ:devices:delta_txs} can be expanded to show the
components of asymmetry $\beta$ of two WR Devices calibrated to the same WR
Calibrator (where $\beta_C$ is the calibrator asymmetry and $\beta_1$,
$\beta_2$ are the internal asymmetries of each device):
\begin{align}
	\Delta_{TX1} = \frac{1}{2}\Delta_1 - \beta_{C1} = \frac{1}{2}\Delta_1 - \beta_1 + \beta_C \\
	\Delta_{RX1} = \frac{1}{2}\Delta_1 + \beta_{C1} = \frac{1}{2}\Delta_1 + \beta_1 - \beta_C \\
	\Delta_{TX2} = \frac{1}{2}\Delta_2 - \beta_{C2} = \frac{1}{2}\Delta_2 - \beta_2 + \beta_C \\
	\Delta_{RX2} = \frac{1}{2}\Delta_2 + \beta_{C2} = \frac{1}{2}\Delta_2 + \beta_2 - \beta_C
\end{align}

After connecting those two WR Devices together, the transmission circuits of
each one communicate with the reception circuits of the other, resulting in a
one-way link delay (without fiber propagation latency):
\begin{align}
	\Delta_{1-2} = \Delta_{TX1} + \Delta_{RX2} = \frac{1}{2}\Delta_1 - \beta_1 + \beta_C + \frac{1}{2} 
		\Delta_2 + \beta_2 - \beta_C  = (\frac{1}{2}\Delta_1 - \beta_1) +
		(\frac{1}{2}\Delta_2 + \beta_2) \\
	\Delta_{2-1} = \Delta_{TX2} + \Delta_{RX1} = \frac{1}{2}\Delta_2 - \beta_2 + \beta_C + \frac{1}{2}
		\Delta_1 + \beta_1 - \beta_C = (\frac{1}{2}\Delta_2 - \beta_2) + 
		(\frac{1}{2}\Delta_1 + \beta_1)
\end{align}

This proves that devices which have been calibrated using the same WR Calibrator
can use the asymmetries found during the calibration process to synchronize one
another.

\subsection{Measurement with a loop-back fiber}
For both measurements the same loop-back fiber, optical transmitter and optical
receiver are used. There is also a requirement in the measurement procedure
(section \ref{subsec:loopback}) saying that both transmitter and receiver should
have a constant delay that doesn't vary for each connection. That means, for
both steps, the loop-back link has some unknown latency $\delta_{L}$.\\

In the first case, the 1-PPS skew measured on the WR Master side can be represented
with the formula:
\begin{equation}
\label{equ:loopback:skew1}
skew_{PPS1} = t_{PPSM1} - (t_{PPSS1} + \delta_{L})
\end{equation}
where $t_{PPSM1}$ is a WR Master absolute time of 1-PPS generation, $t_{PPSS1}$
is a WR Slave absolute time of 1-PPS generation. The latency of the loop-back
fiber $\delta_{L}$ is added to $t_{PPSS1}$, because in the first step the Slave
1-PPS signal observed on the WR Master side is delayed by $\delta_{L}$
picoseconds.\\

In the second step, the situation is reversed. The measurement is made on the WR
Slave side, which means the 1-PPS generated from the WR Master is observed
$\delta_{L}$ picoseconds later:
\begin{equation}
\label{equ:loopback:skew2}
skew_{PPS2} = (t_{PPSM2} + \delta_{L}) - t_{PPSS2}
\end{equation}

The actual $skew_{PPS}$ that we want to measure within this procedure is the
difference between the absolute time of the 1-PPS generation on Master and Slave:
\begin{equation}
\label{equ:lookback:offset}
skew_{PPS} = t_{PPSM1} - t_{PPSS1} = t_{PPSM2} - t_{PPSS2}
\end{equation}
Of course we can make those subtractions equal only because the measurement in
both cases is done when WR Master and WR Slave are synchronized. Now, putting
together equations \ref{equ:loopback:skew1}, \ref{equ:loopback:skew2} and
\ref{equ:lookback:offset} the following system of equations with two unknowns is
produced:
\begin{align}
skew_{PPS1} = skew_{PPS} - \delta_{L}\\
skew_{PPS2} = skew_{PPS} + \delta_{L}
\end{align}
Solving it creates the final formula to calculate the 1-PPS skew between the WR
Master and the WR Slave:
\begin{equation}
skew_{PPS} = \frac{1}{2} (skew_{PPS1} + skew_{PPS2})
\end{equation}

\subsection{Recovering the calibrator}
The new WR Calibrator has unknown transmission and reception delays as any
other, uncalibrated WR Device. We represent them using the mean
(coarse) delay ($\Delta_{C2}$) and the asymmetry factor ($\beta_{C2}$):
\begin{align}
	\Delta_{TXC2} = \frac{1}{2}\Delta_{C2} - \beta_{C2}\\
	\Delta_{RXC2} = \frac{1}{2}\Delta_{C2} + \beta_{C2}
\end{align}

We already know from the previous sections that a WR Device (D1) calibrated to
the primary calibrator (C1) compensates its own asymmetry but also the asymmetry
of the WR Calibrator:
\begin{align}
	\Delta_{TXD1} = \frac{1}{2}\Delta_{D1} - \beta_{D1} + \beta_{C1} \\
	\Delta_{RXD1} = \frac{1}{2}\Delta_{D1} + \beta_{D1} - \beta_{C1}
\end{align}

In an ideal case, when each WR Device knows its delays, the Master-to-Slave
(one-way) delay without the fiber propagation latency would be:
\begin{equation}
\label{equ:recc:delaymsideal}
	\Delta_{D1-C2_{ideal}} = \Delta_{TXD1_{ideal}} + \Delta_{RXC2_{ideal}} = \frac{1}{2}\Delta_{D1} - \beta_{D1} + \frac{1}{2}\Delta_{C2} + \beta_{C2}
\end{equation}

On the other hand, since the WR Device \emph{D1} compensates also the asymmetry
of the primary calibrator \emph{C1} and initially $\beta_{C2}$ is unknown (set
to 0), the actual fixed delay for \emph{D1}-\emph{C2} connection is:
\begin{equation}
\label{equ:recc:delayms}
	\Delta_{D1-C2} = \frac{1}{2}\Delta_{D1} - \beta_{D1} + \beta_{C1} + \frac{1}{2}\Delta_{C2}
\end{equation}

Comparing equations \ref{equ:recc:delayms} and \ref{equ:recc:delaymsideal} it
can be noticed that the factor $\beta_{C1}$ partially compensates the asymmetry
of the new calibrator \emph{C2}. The uncompensated part:
\begin{equation}
	\beta'_{C2} = \beta_{C2} - \beta_{C1}
\end{equation}
produces an additional skew of the 1-PPS signals in the same way as the
uncompensated asymmetry of the WR Device in section \ref{subsec:apx:devices}:
\begin{equation}
	skew_{PPS} = \beta_{C2} - \beta_{C1}
\end{equation}

This remaining asymmetry of the \emph{D1}-\emph{C2} connection is compensated in
the calibration procedure by using the 1-PPS skew as the correction factor.
Then, the transmission and reception delays of the new calibrator \emph{C2} are
presented in the equations:
\begin{align}
	\Delta_{TXC2} = \frac{1}{2}\Delta_{C2} - skew_{PPS} = \frac{1}{2}\Delta_{C2} - \beta_{C2} + \beta_{C1}\\
	\Delta_{RXC2} = \frac{1}{2}\Delta_{C2} + skew_{PPS} = \frac{1}{2}\Delta_{C2} + \beta_{C2} - \beta_{C1}
\end{align}

Each of them has the asymmetry factor $\beta_{C2}$ reduced by $\beta_{C1}$ so
that the actual hardware asymmetry is reduced only partially. The remaining,
uncompensated part equals the asymmetry of the primary calibrator \emph{C1}, so
that the new calibrator \emph{C2} behaves for all practical purposes as the old
calibrator \emph{C1}.
