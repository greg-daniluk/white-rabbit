\section{Equipment requirements}

The calibration procedure described in this document makes no assumption on what 
kind of WR device will be used as a WR Calibrator or what kind of fiber (length, 
producer) is placed in between WR Devices. However, there are a few general 
requirements listed below. The first two of them are essential for the
Calibrator pre-calibration steps described in section
\ref{sec:procedure:calibrator}.\\

WR Calibration needs:
\begin{itemize}
	\item two fiber cables having different lengths, one of them is a few
		kilometers long (to make $\alpha$ coefficient measurement precise), while
		the other is a few meters long.
	\item two pieces of a WR Device selected to be the Calibrators: this means two
		exactly the same WR Devices, having the same PCB design, the same bitstream
		downloaded into the FPGA and using complementary SFP transceivers (the same
		producer, matching Tx/Rx WDM wavelengths\footnote{Check the list of
		supported SFP transceivers http://www.ohwr.org/projects/white-rabbit/wiki/SFP});
	\item all WR Devices (i.e. the Calibrator and devices under calibration)
		capable of producing 1-PPS output from their local clock 
		and providing it through an on-board connector or test point;
	\item two oscilloscope cables of the same delay (i.e. the same type and
		length) or different but known delays so that the correction
		could be made for all oscilloscope measurements described in
		the document;
	\item a device of any producer that is able to measure time
		spans of less than 1ns for measuring 1-PPS skew. This can be an oscilloscope or timer/counter device e.g. Pendulum CNT-91.
\end{itemize}
